\chapter*{Introduction}
\addcontentsline{toc}{chapter}{Introduction}

The process of drug \cite{Bjornsson2019-dq} discovery and clinical development is notoriously time consuming, costly, and sometimes burdened
with higher failure rates, particularly when we plan to make a transition from preclinical studies to human trials. traditional clinical trials, despite their rigor, are often limited by population heterogeneity, ethical concerns, rigid design.
this ineffeciency has sparked global interest in insilico clinical trials (ISTs), where computer-simulated models-especially AI-Powered
digital twins \cite{Viceconti_Henney_Morley-Fletcher_2016} mimic patient-specific biological and pharmacological responses to therapies, enabling faster, safer, and more cost-effective 
decision-making in drug development().

Digital twins \cite{Parekh2019-hq}, originally used in engineering, are now redefining healthcare by enabling real-time, individualized simulations.
when powered by machine learning and system biology, digital twins can integrate multi-omics, clinical, and lifestyle data to simulate disease disease progressionand thrapeutic responses in 
virtual patients. this patient-specific approach, capable of iterating thousands of tria scenarios, offer a paradigm shift on how we test drug efficacy, optimize dosing and stratify resopnders all before
a drug is ever administered to a human. as shown bu (), this reduces the trial costs, enhances precision medicine ,and mitigates adverse outcomes in at-risk subpopulations.

Several recent studies have validated this approach for instance, the Avicenna Allience and the InSilicoTrials platform have demonstrated the regulatory
potential of digital twins, accelerating drug appproval pipelines (). Additionally, () used a digital twin of type 1 diabetes petients to successfully 
simulate glucose responses to insulin thrapies. Similarly, recent research by () highlighted the use of AI and digital twins for predicting 
outcomes in oncology trials. these examples provide a growing evidence base and show that regulatory bodies like the FDA and EMA are increasingly receptive
to ISTs as complementary to traditional trials.

Despite these momentum,  there remain scientific and technical challenges. Building clinically relevant digital twins requires 
robust integration of diverse datasets, validation against real-world evidence, and alignment with ethical and regulatory frameworks. The proposed
project aims to fill these gaps by developing an end-to-end digital twin frameworks that incorporates machine learning, mechanistic modeling and clinical 
trial simulation tailored to specific therapeutic areas.

This project will be highly relevant to both academia and industry stakeholders. Biopharmaceutical companies
seek to reduce the cost and duration of clinical trials, and regulators aim to promote patient safety and innovtion.
through this interdisciplinary approach, combining AI, Bioinformatics, Pharmacokinetics, and Clinical Data Science,
the research is poised to contribute to a new era of AI-enabled precision medicine and more ethical, efficient drug development pipelines.
